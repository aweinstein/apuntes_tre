%%%%%%%%%%%%%%%%%%%%%%%%%%%%%%%%%%%%%%%%%
% Short Sectioned Assignment
% LaTeX Template
% Version 1.0 (5/5/12)
%
% This template has been downloaded from:
% http://www.LaTeXTemplates.com
%
% Original author:
% Frits Wenneker (http://www.howtotex.com)
%
% License:
% CC BY-NC-SA 3.0 (http://creativecommons.org/licenses/by-nc-sa/3.0/)
%
%%%%%%%%%%%%%%%%%%%%%%%%%%%%%%%%%%%%%%%%%
\documentclass[paper=letter, fontsize=11pt]{scrartcl}
\usepackage[spanish,es-noquoting]{babel}
%\usepackage[english]{babel}
\usepackage[T1]{fontenc} % Use 8-bit encoding that has 256 glyphs
\usepackage[utf8]{inputenc}
\usepackage{fourier} % Use the Adobe Utopia font for the document

\usepackage{amsmath,amsfonts,amsthm} % Math packages

\usepackage{sectsty} % Allows customizing section commands
\allsectionsfont{\normalfont\scshape} % Make all sections centered,
                                                 % the default font and small
                                                 % caps
\usepackage[american]{circuitikz}
\usepackage{subfig}
\usepackage{fancyhdr} % Custom headers and footers
\usepackage{enumitem}
\setlist{nolistsep}

\usepackage[pdftex]{hyperref}
\hypersetup{
    colorlinks=true,%
    citecolor=black,%
    filecolor=black,%
    linkcolor=black,%
    urlcolor=blue
}

\pagestyle{fancyplain} % Makes all pages in the document conform to the custom
                       % headers and footers
\fancyhead{} % No page header - if you want one, create it in the same way as
             % the footers below
\fancyfoot[L]{} % Empty left footer
\fancyfoot[C]{} % Empty center footer
\fancyfoot[R]{\thepage} % Page numbering for right footer
\renewcommand{\headrulewidth}{0pt} % Remove header underlines
\renewcommand{\footrulewidth}{0pt} % Remove footer underlines
\setlength{\headheight}{13.6pt} % Customize the height of the header

% \numberwithin{equation}{section} % Number equations within sections (i.e. 1.1,
%                                  % 1.2, 2.1, 2.2 instead of 1, 2, 3, 4)
% \numberwithin{figure}{section} % Number figures within sections (i.e. 1.1, 1.2,
%                                % 2.1, 2.2 instead of 1, 2, 3, 4)
% \numberwithin{table}{section} % Number tables within sections (i.e. 1.1, 1.2,
%                               % 2.1, 2.2 instead of 1, 2, 3, 4)


\newcommand{\horrule}[1]{\rule{\linewidth}{#1}} % Create horizontal rule
                                                % command with 1 argument of
                                                % height

\title{
\normalfont \normalsize
\textsc{Universidad de Valparaíso, Departamento de Ingeniería Biomédica} \\
[25pt]
%\horrule{0.5pt} \\[0.4cm] % Thin top horizontal rule
\huge Teoría de Redes \\ % The assignment title
\LARGE Repaso
%\horrule{2pt} \\[0.5cm] % Thick bottom horizontal rule
}

\author{Alejandro Weinstein}

\date{\normalsize 1\ts{er} semestre 2013} % Today's date or a custom date

\input{../macros.tex}

\begin{document}

\maketitle

Lo siguiente es material visto anteriormente en la asignatura Bases de
Electromedicina.

\section{Voltaje, corriente y potencia}

Las dos cantidades fundamentales en el estudio de las redes eléctricas son el
\emph{voltaje}, también llamado \emph{tensión} o \emph{potencial eléctrico}, y
la corriente. El voltaje se mide en \emph{volts} y la corriente en
\emph{amperes}. Estas unidades se abrevian como $[V]$ y $[A]$,
respectivamente. El voltaje es una cantidad que se define entre dos puntos en
el espacio; siempre hablamos del ``voltaje entre dos puntos''. La corriente es
una cantidad que se define usando la sección transversal de un elemento;
siempre hablamos de la ``corriente a través de un elemento''. El voltaje se
denota por $V$ cuando es constante, o por $v$ cuando es una función en el
tiempo. Análogamente, la corriente se denota por $I$ cuando es constante, o por
$i$ cuando es función del tiempo. En ocasiones, para enfatizar la dependencia
del tiempo, el voltaje y la corrientes se denotan por $v(t)$ e $i(t)$,
respectivamente.

La otra cantidad relevante es la \emph{potencia eléctrica}. Esta cantidad está
dada por el producto del voltaje y la corriente. Dependiendo de la polaridad
con que se definen el voltaje y la corriente en un elemento, y del signo de la
potencia, la potencia es disipada o entregada por el elemento. La potencia se
mide en \emph{watt}; esta unidad se abrevia como $[W]$. La potencia se denota
por $P$ cuando es constante, o por $p$ cuando es una función del tiempo. Al
igual que para el voltaje y la corriente, para enfatizar la dependencia del
tiempo, en ocasiones la potencia se denota por $p(t)$. En resumen, podemos
escribir
%
\begin{gather}
  \label{eq:pot}
  P = V I \\
  p(t) = v(t) i(t).
\end{gather}

%TODO
% conexion a tierra

\section{Dispositivos  eléctricos}

Un \emph{circuito eléctrico}, o simplemente un \emph{circuito}, es un conjunto
de elementos eléctricos interconectados entre sí. En esta asignatura estudiamos
circuitos formados por la interconexión de componentes de dos
terminales.\footnote{Existen componentes electrónicos de más de dos terminales,
  \eg, transistores, amplificadores operacionales, \etc, pero estos componentes
  están fuera del alcance del curso. Uno también puede considerar los
  transformadores como dispositivos de cuatro terminales. Por simplicidad
  asumiremos que los transformadores son inductores acoplados magnéticamente.}
Estos dispositivos son: resistencias, condensadores, inductores, fuentes de
voltaje, y fuentes de corriente. La figura \ref{fig:RLCsymb} muestra los símbolos
respectivos.

\begin{figure}[h!]
  \centering
  \begin{circuitikz}
    \draw (0,0) to[R, v=$v_R$, i>^=$i_R$, l=$R$, *-*] (2.5,0);
    \draw (3,0) to[C, v=$v_C$, i>^=$i_C$, l=$C$, *-*] (5.5,0);
    \draw (6,0) to[L, v=$v_L$, i>^=$i_L$, l=$L$, *-*] (8.5,0);
  \end{circuitikz}
  \caption{Símbolos de resistencia, condensador e inductor.}
\label{fig:RLCsymb}
\end{figure}

Los dispositivos eléctricos de dos terminales se caracterizan por la relación
voltaje-corriente entre sus terminales. Las siguientes ecuaciones describen
estas relaciones para las resistencias, condensadores, e inductancias,
respectivamente:
%
\begin{gather}
  v_R = R i_R   \label{eq:Rvi}\\
  i_C = C \frac{dv_c}{dt} \label{eq:Cvi}\\
  v_L = L \frac{di_L}{dt} \label{eq:Lvi}.
\end{gather}
%
Las constantes $R$, $L$, y $C$ que aparecen en estas ecuaciones son parámetros
cuyo valor depende del elemento en particular que se utilice. La constante $R$
se conoce como resistencia y se mide en \emph{ohms}. La constante $C$ se conoce
como capacitancia y se mide en \emph{faradio}. La constante $L$ se conoce como
inductancia y se mide en \emph{henry}. Estas unidades se abrevian usando los
símbolos $[\Omega]$, $[F]$, y $[L]$, respectivamente. La ecuación
\eqref{eq:Rvi} se conoce como \emph{Ley de Ohm}. Las relaciones
voltaje-corriente para los condensadores e inductores \eqref{eq:Cvi} y
\eqref{eq:Lvi} también se pueden escribir como
%
\begin{gather}
  v_C(t) = \frac{1}{C} \int_0^t i_c(x) \ dx + v_c(0) \label{eq:Civ}\\
  i_L(t) = \frac{1}{L} \int_0^t v_L(x) \ dx + i_L(0). \label{eq:Liv}
\end{gather}

La potencia disipada en una resistencia se puede calcular fácilmente combinando
la definición de potencia eléctrica dada por la ecuación \eqref{eq:pot} con la
ley de Ohm:
%
\begin{equation}
  \label{eq:JouleLoss}
  p_R = v_R i_R = R i_R^2 = \frac{V_R^2}{R}.
\end{equation}

Los condensadores e inductores son elementos que almacenan energía. Los
condensadores almacenan energía en el campo eléctrico, mientras que los
inductores lo hacen en el campo magnético. La energía almacenada está dada por
%
\begin{gather}
  \label{eq:LCener}
  E_C = \frac{1}{2} C v_C^2 \\
  E_L = \frac{1}{2} L i_L^2,
\end{gather}
%
donde $E_C$ y $E_L$ son la energía almacenada en un condensador de capacitancia
$C$ faradios y un inductor de inductancia $L$ henrys, respectivamente.

Las fuente de voltaje definen un voltaje entre sus terminales que no depende de
la corriente que pasa a través de ellas. Análogamente, las fuentes de corriente
definen una corriente a través de ellas que no dependen del voltaje entre sus
terminales. La figura \ref{fig:sourceSymb} muestra los símbolos
correspondientes.

\begin{figure}[h!]
  \centering
  \begin{circuitikz}
    \draw (0,0) to[V=$V_s$, *-*] (0,2);
    \draw (2,0) to[sV=$v_s$, *-*] (2,2);
    \draw (4,0) to[I=$I_s$, *-*] (4,2);
    \draw (6,0) to[sI=$i_s$, *-*] (6,2);
  \end{circuitikz}
  \caption{Símbolos de las fuentes independientes de voltaje y corriente.}
\label{fig:sourceSymb}
\end{figure}

Para modelar ciertos dispositivos electrónicos, \eg transistores y
amplificadores operacionales, es útil definir las fuentes
dependientes---también llamadas fuentes controladas. Las fuente de voltaje
dependiente definen un voltaje entre sus terminales que no depende de la
corriente que pasa a través de ellas, y que es función de un voltaje o
corriente definido en el circuito. Análogamente, las fuentes de corriente
definen una corriente a través de ellas que no dependen del voltaje entre sus
terminales, y que es función de un voltaje o corriente definido en el
circuito. La figura \ref{fig:depSourceSymb} muestra los símbolos
correspondientes.


\begin{figure}[h!]
  \centering
  \begin{circuitikz}
    \draw (0,0) to[cV=$V_s$, i>^=$i$, *-*] (0,2);
    \draw (2,0) to[csV=$v_s$, i>^=$i$, *-*] (2,2);
    \draw (4,0) to[cI=$I_s$, *-*] (4,2);
    \draw (6,0) to[csI=$i_s$, *-*] (6,2);
  \end{circuitikz}
  \caption{Símbolos de las fuentes dependientes de voltaje y corriente.}
\label{fig:depSourceSymb}
\end{figure}

% TODO:
% Symbols
% Note about dependent independent sources
% Passive vs active
% Lumped vs distributed
% Energia almacenada en un condensador/inductor


\section{Leyes de circuitos}

El problema central en el análisis de redes eléctricas es, dado un circuito,
encontrar el valor del voltaje y corriente correspondiente a cada
elemento. Además de satisfacer la relación voltaje-corriente para cada
elemento, los voltajes y corrientes en el circuito deben satisfacer las
\emph{leyes de Kirchhoff}.

Para todo nodo de un circuito, la ley de corriente de Kirchhoff (LCK) establece
que
%
\begin{equation}
  \label{eq:LCK}
  \sum_{k=1}^n i_k = 0,
\end{equation}
%
donde $i_k$ es la $k$-ésima corriente que entra al nodo, y $n$ es el número de
ramas que tiene el nodo. Esta ecuación asume que las corrientes están definidas
ya sea como todas entrando al nodo, o todas saliendo del nodo. Si una corriente
está definida en el sentido opuesto, debe entrar con signo negativo en la
sumatoria.

Para todo lazo cerrado de un circuito, la ley de voltaje de Kirchhoff (LVK)
establece que
%
\begin{equation}
  \label{eq:LVK}
  \sum_{k=1}^n v_k = 0,
\end{equation}
%
donde $v_k$ es el $k$-ésimo voltaje en el lazo, y $n$ es el número de elementos
en el lazo. Al igual que para la LCK, esta ecuación requiere ser consistente
con los signos de los voltajes.

\begin{ex}
  Encuentre $v_R$ en el circuito que sigue.

\begin{figure}[h!]
  \centering
  \begin{circuitikz} \draw
    (0,0) to[V=$V$] (0,2)
    to[short] (2,2)
    to[R, v=$v_R$] (2,0)
    to[short](0,0);
  \end{circuitikz}
\end{figure}

Aplicando LVK se tiene
%
\begin{equation*}
  V - v_R = 0 \Rightarrow v_R = V.
\end{equation*}
\end{ex}


\begin{ex}[Divisor de voltaje]
  Encuentre $v_{R2}$ en el circuito que sigue.

\begin{figure}[h!]
  \centering
  \begin{circuitikz} \draw
    (0,-1) to[V=$V$, i=$i$] (0,3)
    to[short] (2,3)
    to[R=$R_1$, v=$v_{R1}$] (2,1)
    to[R=$R_2$, v=$v_{R2}$] (2,-1)
    to[short](0,-1)
  ;\end{circuitikz}
\end{figure}

Aplicando LVK y ley de Ohm se tiene
%
\begin{gather*}
  V - v_{R1} - v_{R2} = 0 \Rightarrow v_{R1} + v_{R2} = V
  \Rightarrow R_1 i + R_2 i = V \Rightarrow i = \frac{V}{R_1 + R_2}\\
  \Rightarrow v_{R2} = \frac{R_2}{R_1 + R_2} V.
\end{gather*}
\end{ex}

\begin{ex}[Resistencia en serie]
  Para el circuito que sigue, encuentre la resistencia equivalente entre los
  terminales $A$ y $B$.

\begin{figure}[h!]
  \centering
  \begin{circuitikz} \draw
    (0,4) node[anchor=east] {$A$}
    to[short, o-, i=$i$] (1,4)
    to[R=$R_1$, v=$v_{R1}$] (1,2)
    to[R=$R_2$, v=$v_{R2}$] (1,0)
    to[short, -o] (0,0) node[anchor=east] {$B$}
  ;\end{circuitikz}
\end{figure}

Aplicando LVK y ley de Ohm se tiene
\begin{equation*}
R_1 i + R_2 i = v \Rightarrow R_{eq} = \frac{v_{AB}}{i} = R_1 + R_2.
\end{equation*}
\end{ex}

\begin{ex}[Resistencia en paralelo]
  Para el circuito que sigue, encuentre la resistencia equivalente entre los
  terminales $A$ y $B$.

\begin{figure}[h!]
  \centering
  \begin{circuitikz} \draw
    (0,3) node[anchor=east] {$A$}
    to[short, o-, i=$i$] (1,3)
    to[R=$R_1$, v=$v_{R1}$, i>^=$i_1$] (1,0)
    to[short, -o] (0,0) node[anchor=east] {$B$}
    (1,3) to[short] (3,3)
    to[R=$R_2$, v=$v_{R2}$, i>^=$i_2$] (3,0)
    to[short] (1,0)
  ;\end{circuitikz}
\end{figure}

Aplicando LVK, LCK y ley de Ohm se tiene
\begin{equation*}
  v_{AB} = v_{R1} = v_{R2};\ i = i_1 + i_2 = \frac{V_{AB}}{R_1} +
  \frac{V_{AB}}{R_1} \ \Rightarrow
  R_{eq} = \frac{v_{AB}}{i} = \left(\frac{1}{R_1} + \frac{1}{R_2} \right)^{-1}
  = \frac{R_1 R_2}{R_1 + R_2} := R_1 \parallel R_2.
\end{equation*}
\end{ex}

% TODO:
% Ejemplos
% Divisor de corriente

\section{Métodos de análisis}

En un circuito con sólo fuentes y resistencias, la combinación de la
característica voltaje-corriente de todos los elementos junto con las leyes de
Kirchhoff, da paso a un sistema de ecuaciones lineales.\footnote{Si además de
  resistencias el circuito contiene condensadores o inductores, uno termina con
  un sistema de ecuaciones diferenciales.} Los siguientes son los dos métodos
usados comúnmente para encontrar estas ecuaciones.

El Método de Corrientes de Malla (MCM) aplica la LVK en cada malla del
circuito. Para cada malla se define una corriente que completa un lazo
cerrado. Aunque no es estrictamente necesario, es conveniente definir todas las
corrientes de malla en el sentido de las agujas del reloj. Cuando una rama del
circuito es compartida por dos mallas, la corriente a través de esa rama es la
suma---considerando el signo---de las corrientes de malla respectivas. Una vez
que se definen las corrientes, se aplica la LVK para cada rama, obteniéndose un
sistema de $n$ ecuaciones lineales, donde $n$ es el número de mallas en el
circuito.

\begin{ex}[Método de corriente de malla] Para el circuito que sigue y usando
  MCM, encontrar las ecuaciones necesarias para encontrar todas las corrientes
  de malla.

  \begin{figure}[h!]
    \centering
    \begin{circuitikz}
      %\draw[help lines] (0,0) grid (6,3);
      \draw (0,0) to[V=$V_A$] (0,3)
       to[R=$R_1$, v<=$v_{R1}$] (3,3)
       to[R=$R_3$, v=$v_{R3}$] (6,3);
       \draw (6,0) to[V_=$V_B$] (6,3);
       \draw (0,0) to[short] (6,0);
       \draw (3,0) to[R=$R_2$, v>=$v_{R2}$] (3,3);
       \draw [thick, <-] (2,1.5) arc(0:270:0.70);
       \draw (1.4,1.4) node {$i_1$};
       \draw [thick, <-] (5.4,1.5) arc(0:270:0.70);
       \draw (4.8,1.4) node {$i_2$};
    \end{circuitikz}
  \end{figure}
\end{ex}

Aplicando LVK y ley de Ohm en cada malla se obtiene
%
\begin{gather*}
  v_A - v_{R1} - v_{R2} = 0 \Rightarrow v_A - R_1i_1 - R_2(i_1 - i_2) = 0 \\
  v_{R2} - v_{R3} - v_B = 0 \Rightarrow R_2(i_1 - i_2) - R_3i_2 - v_B = 0.
\end{gather*}

El Método de Voltaje de Nodos (MVN) aplica la LCK en los nodo del circuitos. En
forma arbitraria se define uno de los nodos del circuito como nodo de
referencia. Luego, para cada uno de los nodos restantes se define un voltaje
entre el nodo y el nodo de referencia. Usando estos voltajes, se aplica la LCK
en cada nodo, obteniéndose un sistema de $n - 1$ ecuaciones lineales, donde $n$
es el número de nodos en el circuito.

\begin{ex}[Método de voltaje de nodo] Para el circuito que sigue y usando MVN,
  encontrar las ecuaciones necesarias para encontrar todos los voltajes de
  nodo.

  \begin{figure}[h!]
    \centering
    \begin{circuitikz}
      %\draw[help lines] (0,0) grid (6,3);
      \draw (0,0) to[V=$V_A$, *-*] (0,3)
       to[R=$R_1$, *-*] (3,3)
       to[R=$R_3$, *-*] (6,3)
       to[R=$R_5$, *-*] (9,3);
       \draw (3,0) to[R=$R_2$, *-*] (3,3);
       \draw (6,0) to[R=$R_4$, *-*] (6,3);
       \draw (0,0) to[short] (9,0);
       \draw (9,0) to[V=$V_B$, *-*] (9,3);
       \draw {[above] (0,3) node {$v_1$} (3,3) node {$v_2$}
         (6,3) node {$v_3$} (9,3) node {$v_4$}};
       \draw (3,0) node[below] {$v_{ref}$};
    \end{circuitikz}
  \end{figure}

Aplicando LCK y ley de Ohm en cada nodo, y LVK en los casos particulares de
$v_1$ y $v_4$ se obtiene
%
\begin{gather*}
  \frac{v_2 - v_1}{R_1} + \frac{v_2}{R_2} + \frac{v_2 - v_3}{R_3} = 0 \\
  \frac{v_3 - v_2}{R_3} + \frac{v_3}{R_4} + \frac{v_3 - v_4}{R_5} = 0 \\
  v_1 = V_A \\
  v_4 = V_B.
\end{gather*}
\end{ex}

\section{Otros conceptos}

\subsection{El principio de superposición}

Una propiedad de los sistemas lineales en general, y de los circuitos lineales
en particular, es que las variables del sistema pueden ser calculadas como la
suma de los efectos de cada variable independiente. En el caso de los circuitos
lineales, esto se traduce en que los voltajes y corrientes se pueden calcular
como la suma de los efectos de las fuentes independientes. Esta propiedad se
conoce como \emph{principio de superposición}. Para usar este principio, se
calcula el efecto de cada fuente independiente individualmente, con el resto de
las fuentes ``apagadas''. Una fuente se considera como ``apagada'' cuando la
variable asociada a ésta es igual a cero. En el caso de una fuente de voltaje
independiente, esto es equivalente a reemplazar la fuente por un
cortocircuito. En el caso de una fuente de corriente independiente, esto es
equivalente a reemplazar la fuente por un circuito abierto.

\begin{ex}[Principio de superposición] Usando el principio de superposición,
  encuentre el voltaje $v_{R2}$.

  \begin{figure}[h!]
    \centering
    \begin{circuitikz}
      \draw (0,0) to[V=$V_A$] (0,3)
       to[R=$R_1$] (3,3)
       to[R=$R_3$] (6,3);
       \draw (6,0) to[V_=$V_B$] (6,3);
       \draw (0,0) to[short] (6,0);
       \draw (3,0) to[R=$R_2$, v>=$v_{R2}$] (3,3);
    \end{circuitikz}
  \end{figure}

  \begin{enumerate}[label=\roman*]
  \item Efecto de $V_A$ ($V_B$ apagada)

  El circuito que sigue corresponde a esta condición.  Con $V_B$ apagada, las
  resistencias $R_2$ y $R_3$ quedan en paralelo, y el efecto del voltaje $V_A$
  sobre $v_{R2}$ se puede calcular usando un divisor de voltaje.

  \begin{figure}[h!]
    \centering
    \begin{circuitikz}
      \draw (0,0) to[V=$V_A$] (0,3)
       to[R=$R_1$,] (3,3)
       to[R=$R_3$] (6,3);
       \draw (6,0) to[short] (6,3);
       \draw (0,0) to[short] (6,0);
       \draw (3,0) to[R=$R_2$, v>=$v^A_{R2}$] (3,3);
    \end{circuitikz}
  \end{figure}

  \begin{equation*}
    v^A_{R2} = \frac{R_2 \parallel R_3}{ (R_2 \parallel R_3) + R_1} V_A.
  \end{equation*}

  \item Efecto de $V_B$ ($V_A$ apagada)

    El circuito que sigue corresponde a esta condición. Con $V_A$ apagada, las
    resistencias $R_1$ y $R_2$ quedan en paralelo, y el efecto del voltaje
    $V_B$ sobre $v_{R2}$ se puede calcular usando un divisor de voltaje.

  \begin{figure}[h!]
    \centering
    \begin{circuitikz}
      \draw (0,0) to[short] (0,3)
       to[R=$R_1$] (3,3)
       to[R=$R_3$] (6,3);
       \draw (6,0) to[V_=$V_B$] (6,3);
       \draw (0,0) to[short] (6,0);
       \draw (3,0) to[R=$R_2$, v>=$v^B_{R2}$] (3,3);
    \end{circuitikz}
  \end{figure}

  \begin{equation*}
    v^B_{R2} = \frac{R_1 \parallel R_2}{ (R_1 \parallel R_2) + R_3} V_B.
  \end{equation*}
  \end{enumerate}

Finalmente, el voltaje $v_{R2}$ se calcula como la superposición de los dos
voltajes anteriores:
%
\begin{equation*}
  v_{R2} = v^A_{R2} + v^B_{R2} = \frac{R_2 \parallel R_3}{ (R_2 \parallel R_3)
    + R_1} V_A + \frac{R_1 \parallel R_2}{ (R_1 \parallel R_2) + R_3} V_B.
\end{equation*}
\end{ex}


\subsection{Equivalente Thévenin}

El \emph{equivalente Thévenin} permite representar una red de múltiples
elementos por tan sólo una fuente de voltaje en serie con una resistencia. Dado
un circuito, esta red equivalente se encuentra realizando los siguientes
paso. Primero se definen dos subredes $A$ y $B$, tal que estas subredes están
conectadas a través de un par de terminales. Defina el voltaje de terminales
abiertos $v_{th}$ como el voltaje que aparecería entre estos dos terminales de
la red $A$ si la red $B$ se desconectara. Luego ``apague'' todas la fuentes
independientes---es decir, reemplace todas la fuentes independientes de voltaje
por un cortocircuito, y todas las fuentes independientes de corriente por un
circuito abierto---y calcule la resistencia equivalente entre los dos
terminales de la red $A$ bajo estas condiciones. Llame a esta resistencia
equivalente $R_{th}$. La red Thévenin equivalente está dada por una fuente de
voltaje de valor $V_{th}$ en serie con una resistencia de valor $R_{th}$.

\begin{ex}[Equivalente Thévenin] Encuentre el equivalente Thévenin de la red
  conectada a la izquierda de los terminales $T_1$ y $T_2$.

  \begin{figure}[h!]
    \centering
    \begin{circuitikz}
      \draw (0,0) to[V=$V_A$] (0,3)
       to[R=$R_1$,] (3,3)
       to[R=$R_2$] (3,0)
       to[short] (0,0);
      \draw (3,3) to[short, -*] (4.5,3) node[above] {$T_1$}
      to[short] (6,3)
      to[R=$R_L$] (6,0)
      to[short, -*] (4.5,0) node[below] {$T_2$}
      to[short] (3,0);
    \end{circuitikz}
  \end{figure}

Si se desconecta $R_2$, el voltaje entre los terminales $T_1$ y $T_2$ está dado
por el divisor de voltaje
%
\begin{equation*}
  v_{th} = \frac{R_2}{R_1 + R_2} V_A.
\end{equation*}
%
Al apagar la fuente de voltaje $V_A$, la resistencia equivalente de la red
conectada a la izquierda de los terminales está dada por la conexión en
paralelo de $R_1$ y $R_2$:
%
\begin{equation*}
  R_{th} = R_1 \parallel R_2.
\end{equation*}
\end{ex}


% \section{Matemáticas}

% Los siguientes son algunos de los conceptos matemáticos utilizados en el
% análisis de redes eléctricas.

% \subsection{Números complejos}

% Un numero o variable compleja se define como $z = a + b j$, donde\footnote{En
%   matemáticas normalmente se define $i:=\sqrt{-1}$. Dado que en teoría de redes
%   la variable $i$ se usa para denotar la corriente, usamos $j$ en vez de $i$.}
% $j:=\sqrt{-1}$ y $a,b \in \real$. Denotamos por $\complex$ al conjunto de
% números complejos, \ie, $z \in \complex$. Esta forma de representar $z$ se
% conoce como forma rectangular. La variable compleja $z \in \complex$ se puede
% escribir en coordenadas polares, también conocida como forma de
% \emph{Stenmetz}, como
% %
% \begin{equation}
%   \label{eq:polarZ}
%   z = r \angle \theta,
% \end{equation}
% %
% donde
% %
% \begin{gather}
%   r = \sqrt{a^2 + b^2}\label{eq:rAngle}\\
%   \theta = \arctan{\frac{b}{a}}.
% \end{gather}

% La fórmula de Euler $e^{j\theta} = \cos \theta + j \sin \theta$ permite
% representar la variable compleja $z$ como
% %
% \begin{equation}
%   \label{eq:zExp}
%   z = r \cos \theta + j r \sin \theta = r e^{j\theta}.
% \end{equation}

% La suma de $z_1 = a_1 + b_1 j$  y $z_2 = a_2 + b_2 j$ está dada por
% %
% \begin{equation}
%   \label{eq:complexSum}
%   z_1 + z_2 = (a_1 + a_2) + (b_1 + b_2) j.
% \end{equation}
% %
% El producto de $z_1 = r_1 \angle \theta_1j$ y $z_2 = r_2 \angle \theta_2$ está
% dada por
% %
% \begin{equation}
%   \label{eq:complexProd}
%   z_1  z_2 = (r_1  r_2) \angle (\theta_1 + \theta_2).
% \end{equation}

% \subsection{Cálculo}

% \subsection{Matrices}

%303-995-4540

\end{document}

%%% Local Variables:
%%% mode: latex
%%% TeX-master: t
%%% Local IspellDict: castellano
%%% End:
