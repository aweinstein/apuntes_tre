%%%%%%%%%%%%%%%%%%%%%%%%%%%%%%%%%%%%%%%%%
% Short Sectioned Assignment
% LaTeX Template
% Version 1.0 (5/5/12)
%
% This template has been downloaded from:
% http://www.LaTeXTemplates.com
%
% Original author:
% Frits Wenneker (http://www.howtotex.com)
%
% License:
% CC BY-NC-SA 3.0 (http://creativecommons.org/licenses/by-nc-sa/3.0/)
%
%%%%%%%%%%%%%%%%%%%%%%%%%%%%%%%%%%%%%%%%%
\documentclass[paper=letter, fontsize=11pt]{scrartcl}
\usepackage[spanish,es-noquoting]{babel}
%\usepackage[english]{babel}
\usepackage[T1]{fontenc} % Use 8-bit encoding that has 256 glyphs
\usepackage[utf8]{inputenc}
\usepackage{fourier} % Use the Adobe Utopia font for the document

\usepackage{amsmath,amsfonts,amsthm} % Math packages

\usepackage{sectsty} % Allows customizing section commands
\allsectionsfont{\normalfont\scshape} % Make all sections centered,
                                                 % the default font and small
                                                 % caps
\usepackage[american]{circuitikz}
\usetikzlibrary{matrix}

\usepackage{subfig}
\usepackage{fancyhdr} % Custom headers and footers
\usepackage{enumitem}
\setlist{nolistsep}

\usepackage[pdftex]{hyperref}
\hypersetup{
    colorlinks=true,%
    citecolor=black,%
    filecolor=black,%
    linkcolor=black,%
    urlcolor=blue
}

\pagestyle{fancyplain} % Makes all pages in the document conform to the custom
                       % headers and footers
\fancyhead{} % No page header - if you want one, create it in the same way as
             % the footers below
\fancyfoot[L]{} % Empty left footer
\fancyfoot[C]{} % Empty center footer
\fancyfoot[R]{\thepage} % Page numbering for right footer
\renewcommand{\headrulewidth}{0pt} % Remove header underlines
\renewcommand{\footrulewidth}{0pt} % Remove footer underlines
\setlength{\headheight}{13.6pt} % Customize the height of the header

% \numberwithin{equation}{section} % Number equations within sections (i.e. 1.1,
%                                  % 1.2, 2.1, 2.2 instead of 1, 2, 3, 4)
% \numberwithin{figure}{section} % Number figures within sections (i.e. 1.1, 1.2,
%                                % 2.1, 2.2 instead of 1, 2, 3, 4)
% \numberwithin{table}{section} % Number tables within sections (i.e. 1.1, 1.2,
%                               % 2.1, 2.2 instead of 1, 2, 3, 4)


\newcommand{\horrule}[1]{\rule{\linewidth}{#1}} % Create horizontal rule
                                                % command with 1 argument of
                                                % height

\title{
\normalfont \normalsize
\textsc{Universidad de Valparaíso, Departamento de Ingeniería Biomédica} \\
[25pt]
%\horrule{0.5pt} \\[0.4cm] % Thin top horizontal rule
\huge Teoría de Redes \\ % The assignment title
\LARGE Circuitos de 1\ts{er} Orden
%\horrule{2pt} \\[0.5cm] % Thick bottom horizontal rule
}

\author{Alejandro Weinstein}

\date{\normalsize 1\ts{er} semestre 2013} % Today's date or a custom date

\input{../macros.tex}

\begin{document}

\maketitle

\section{Descarga de un condensador}

El circuito que sigue muestra un condensador de valor $C$ con condición inicial
$v_C(0)=V_0$, descargándose a través de una resistencia de valor $R$.

\begin{figure}[h!]
  \centering
  \begin{circuitikz}
    \draw (0,0) to[C=$C$, v>=$v_C$, i<=$i_c$] (0,3)
    to[short] (3,3)
    to[R=$R$, i>^=$i_R$, v=$v_R$] (3,0)
    to[short] (0,0);
  \end{circuitikz}
  \caption{Descarga de un condensador.}
\end{figure}

Por LCK y LVK, $i_c = - i_R$ y $v_C = v_R$, respectivamente. Además, $i_c = C
\frac{dv_C}{dt}$ e $i_R = \frac{v_R}{R}$. Combinando estas relaciones obtenemos
%
\begin{gather}
  C\frac{dv_C}{dt} + \frac{v_C}{R} = 0 \\
  \frac{dv_C}{dt} + \frac{v_C}{RC} = 0.   \label{eq:Cdesc}
\end{gather}
%
La ecuación~\eqref{eq:Cdesc}, junto a la condición inicial $v_C(0)=V_0$, es una
ecuación diferencial lineal de 1\ts{er} orden. Resolver esta ecuación equivale
a encontrar una función en el tiempo $v_C(t)$, $t>=0$, que satisfaga
simultáneamente la condición inicial y la ecuación~\eqref{eq:Cdesc}. En otras
palabras, para encontrar como evoluciona el voltaje en el condensador en la
medida que el tiempo transcurre, debemos resolver esta ecuación diferencial.

En este caso resolvemos la ecuación diferencial ejecutando los siguientes
pasos. Primero asumimos que el voltaje en el condensador es de la forma $v_C(t)
= Ae^{st}$, con $A$ y $s$ constantes de valor desconocido. Luego reemplazamos
esta expresión en la ecuación que queremos resolver. Combinando este resultado
con la condición inicial, encontramos los valores de $A$ y $s$.

Reemplazando $v_C(t) = Ae^{st}$ en la ecuación~\eqref{eq:Cdesc} obtenemos
%
\begin{equation}
  \begin{split}
    sA e^{st} + \frac{A}{RC} e^{st} &= 0 \\
    A \left(s + \frac{1}{RC} \right) e^{st} &= 0.
  \end{split}
\end{equation}
%
Como $e^{st}$ es mayor que cero, y $A$ tampoco puede ser cero porque en ese
caso no se podría satisfacer la condición inicial, la única forma de satisfacer
esta ecuación es con $s = -\frac{1}{RC}$. Como $v_C(0) = A$, también podemos
concluir que para satisfacer la condición inicial debemos fijar $A=V_0$. En
resumen,
%
\begin{equation}
  v_C(t) = V_0 e^{-t/RC}, \ t \geq 0.
\end{equation}

En este caso la constante de tiempo de la exponencial decrecientes es
$\tau=RC$, y podemos concluir que al condensador le toma alrededor de $4RC$
segundos descargarse a través de la resistencia. La figura~\ref{fig:Cdesc}
muestra el voltaje en el condensador en función del tiempo.

\begin{figure}[h!]
  \centering
  \begin{tikzpicture}[xscale=1.5, yscale=4]
    \draw [->] (-0.1,0) -- (5.1,0); % x-axis
    \draw [->] (0,-0.1) -- (0,1.2); % y-axis
    \draw (0,1.2) node[right] {$v_C(t)$};
    \draw (5.1,0) node[right] {$t$};
    \draw [blue, ultra thick, domain=0:5, samples=100] plot
    (\x, {exp(-\x)});
    \draw (0,1) node[left] {$V_0$} (4,0) node[below] {$4RC$};
  \end{tikzpicture}
  \caption{Voltaje en un condensador durante su descarga a través de una
    resistencia. }
\label{fig:Cdesc}
\end{figure}

\section{Carga de un condensador}

El circuito que sigue muestra un condensador inicialmente descargado, \ie
$v_C(0)=0$, que se carga a través de la fuente de voltaje continua de valor $V$
en serie con una resistencia de valor $R$.

\begin{figure}[h!]
  \centering
  \begin{circuitikz}
    \draw (0,0) to[V=$V$] (0,3)
    to[R=$R$] (3,3)
    to[C=$C$, i>^=$i$, v=$v_C$] (3,0)
    to[short] (0,0);
  \end{circuitikz}
  \caption{Carga de un condensador.}
\end{figure}

Usando LVK obtenemos
%
\begin{equation}
  R i + v_C = V.
\end{equation}
%
Como $i = C\frac{dv_C}{dt}$, podemos escribir
%
\begin{gather}
  R C \frac{dv_C}{dt} + v_C = V \\
  \frac{dv_C}{dt} + \frac{v_C}{RC} = \frac{V}{RC} \label{eq:CCarg}.
\end{gather}

En este caso la solución de la ecuación diferencial está formada por la
combinación de la solución particular y la solución homogénea. La solución
particular es el voltaje $v_C$ que aparece entre los terminales del condensador
en estado estacionario, en este caso,
%
\begin{equation}
  v_p(t) = V.
\end{equation}
%
La solución homogénea está dada por el voltaje que aparece cuando no hay
fuentes independientes. De la sección anterior sabemos que éste es de la forma
%
\begin{equation}
  v_h(t) = Ae^{-t/RC}.
\end{equation}
%
Combinando la solución particular con la homogénea obtenemos
%
\begin{equation}
  v_c(t) = v_p(t) + v_h(t) = V + A e^{-t/RC}.
\end{equation}
%
Para satisfacer la condición inicial se debe satisfacer que
%
\begin{equation}
  v_C(0) = V + A = 0 \Rightarrow A = -V.
\end{equation}

Finalmente podemos escribir
%
\begin{equation}
  v_c(t) = V - Ve^{-t/RC} \quad t \geq 0.
\end{equation}

La Figura~\ref{fig:Ccarg} muestra el voltaje en el condensador en función del
tiempo. Al igual que durante la descarga, al condensador le toma alrededor de
$4RC$ segundos cargarse a través de la resistencia.

\begin{figure}[h!]
  \centering
  \begin{tikzpicture}[xscale=1.5, yscale=4]
    \draw [->] (-0.1,0) -- (5.1,0); % x-axis
    \draw [->] (0,-0.1) -- (0,1.2); % y-axis
    \draw (0,1.2) node[right] {$v_C(t)$};
    \draw (5.1,0) node[right] {$t$};
    \draw [blue, ultra thick, domain=0:5, samples=100] plot
    (\x, {1 - exp(-\x)});
    \draw [dotted] (0,1) -- (5,1);
    \draw (0,1) node[left] {$V$} (4,0) node[below] {$4RC$};
  \end{tikzpicture}
  \caption{Voltaje en un condensador durante su descarga a través de una
    resistencia. }
\label{fig:Ccarg}
\end{figure}

\section{Descarga de un inductor}

El circuito que sigue muestra un inductor de valor $L$ con condición inicial
$i_L(0)=I_0$, descargándose a través de una resistencia de valor $R$.


Por LVK $v_L + v_R = 0$. Además, $v_L = L \frac{di_L}{dt}$ y $v_R = R
i_L$. Combinando estas relaciones obtenemos
%
\begin{gather}
  L\frac{di_L}{dt} + R i_L = 0 \\
  \frac{di_L}{dt} + \frac{Ri_L}{L} = 0.   \label{eq:Ldesc}
\end{gather}
%

\begin{figure}[h!]
  \centering
  \begin{circuitikz}
    \draw (0,0) to[L=$L$, v>=$v_L$, i<=$i_L$] (0,3)
    to[short] (3,3)
    to[R=$R$, v>=$v_R$] (3,0)
    to[short] (0,0);
  \end{circuitikz}
  \caption{Descarga de un inductor.}
\end{figure}

La ecuación~\eqref{eq:Ldesc}, junto a la condición inicial $i_L(0)=V_0$, es una
ecuación diferencial lineal de 1\ts{er} orden. Al igual que como hicimos para
resolver la ecuación diferencial asociada a la descarga del condensador, en
este caso resolveremos la ecuación diferencial ejecutando los siguientes
pasos. Primero asumimos que la corriente en el inductor es de la forma $i_L(t)
= Ae^{st}$, con $A$ y $s$ constantes de valor desconocido. Luego reemplazamos
esta expresión en la ecuación que queremos resolver. Combinando este resultado
con la condición inicial, encontramos los valores de $A$ y $s$.

Reemplazando $i_L(t) = Ae^{st}$ en la ecuación~\eqref{eq:Ldesc} obtenemos
%
\begin{equation}
  \begin{split}
    sA e^{st} + \frac{AR}{L} e^{st} &= 0 \\
    A \left(s + \frac{R}{L} \right) e^{st} &= 0.
  \end{split}
\end{equation}
%
Como $e^{st}$ es mayor que cero, y $A$ tampoco puede ser cero porque en ese
caso no se podría satisfacer la condición inicial, la única forma de satisfacer
esta ecuación es con $s = -\frac{R}{L}$. Como $i_L(0) = A$, también podemos
concluir que para satisfacer la condición inicial debemos fijar $A=I_0$. En
resumen,
%
\begin{equation}
  i_L(t) = I_0 e^{-Rt/L}, \ t \geq 0.
\end{equation}

En este caso la constante de tiempo de la exponencial decrecientes es
$\tau=L/R$, y podemos concluir que al inductor le toma alrededor de $4L/R$
segundos descargarse a través de la resistencia. La figura~\ref{fig:Ldesc}
muestra el voltaje en el condensador en función del tiempo.

\begin{figure}[h!]
  \centering
  \begin{tikzpicture}[xscale=1.5, yscale=4]
    \draw [->] (-0.1,0) -- (5.1,0); % x-axis
    \draw [->] (0,-0.1) -- (0,1.2); % y-axis
    \draw (0,1.2) node[right] {$i_L(t)$};
    \draw (5.1,0) node[right] {$t$};
    \draw [blue, ultra thick, domain=0:5, samples=100] plot
    (\x, {exp(-\x)});
    \draw (0,1) node[left] {$I_0$} (4,0) node[below] {$4L/R$};
  \end{tikzpicture}
  \caption{Corriente en un inductor durante su descarga a través de una
    resistencia. }
\label{fig:Ldesc}
\end{figure}

\section{Carga de un inductor}

El circuito que sigue muestra un inductor inicialmente descargado, \ie
$v_L(0)=0$, que se carga a través de la fuente de voltaje continua de valor $V$
en serie con una resistencia de valor $R$.

\begin{figure}[h!]
  \centering
  \begin{circuitikz}
    \draw (0,0) to[V=$V$] (0,3)
    to[R=$R$] (3,3)
    to[L=$L$, i>^=$i_L$, v=$v_L$] (3,0)
    to[short] (0,0);
  \end{circuitikz}
  \caption{Carga de un inductor.}
\end{figure}

Usando LVK obtenemos
%
\begin{equation}
  v_L + R i_L  = V.
\end{equation}
%
Como $v_L = L\frac{di_L}{dt}$, podemos escribir
%
\begin{gather}
  L \frac{di_L}{dt} + R i_L = V \\
  \frac{di_L}{dt} + \frac{R i_L}{L} = \frac{V}{L}. \label{eq:LCarg}
\end{gather}

Al igual que el caso de la carga del condensador, la solución de la ecuación
diferencial está formada por la combinación de la solución particular y la
solución homogénea. La solución particular es la corriente $i_L$ que aparece a
través del inductor en estado estacionario, en este caso,
%
\begin{equation}
  i_p(t) = \frac{V}{R}.
\end{equation}
%
La solución homogénea está dada por la corriente que aparece cuando no hay
fuentes independientes. De la sección anterior sabemos que éste es de la forma
%
\begin{equation}
  i_h(t) = Ae^{-Rt/L}.
\end{equation}
%
Combinando la solución particular con la homogénea obtenemos
%
\begin{equation}
  i_L(t) = i_p(t) + i_h(t) = \frac{V}{R} + A e^{-Rt/L}.
\end{equation}
%
Para satisfacer la condición inicial se debe cumplir
%
\begin{equation}
  i_L(0) = \frac{V}{R} + A = 0 \Rightarrow A = -\frac{V}{R}.
\end{equation}

Finalmente podemos escribir
%
\begin{equation}
  i_L(t) = \frac{V}{R} - \frac{V}{R}e^{-Rt/L} \quad t \geq 0.
\end{equation}

La Figura~\ref{fig:Lcarg} muestra la corriente en el inductor en función del
tiempo. Al igual que durante la descarga, al inductor le toma alrededor de
$4L/R$ segundos cargarse a través de la resistencia.

\begin{figure}[h!]
  \centering
  \begin{tikzpicture}[xscale=1.5, yscale=4]
    \draw [->] (-0.1,0) -- (5.1,0); % x-axis
    \draw [->] (0,-0.1) -- (0,1.2); % y-axis
    \draw (0,1.2) node[right] {$i_L(t)$};
    \draw (5.1,0) node[right] {$t$};
    \draw [blue, ultra thick, domain=0:5, samples=100] plot
    (\x, {1 - exp(-\x)});
    \draw [dotted] (0,1) -- (5,1);
    \draw (0,1) node[left] {$\frac{V}{R}$} (4,0) node[below] {$4L/R$};
  \end{tikzpicture}
  \caption{Corriente en un inductor durante su descarga a través de una
    resistencia. }
\label{fig:Lcarg}
\end{figure}

\section{Análisis en estado estacionario}

\end{document}

%%% Local Variables:
%%% mode: latex
%%% TeX-master: t
%%% Local IspellDict: castellano
%%% End:
