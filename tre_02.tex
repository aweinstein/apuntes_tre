%%%%%%%%%%%%%%%%%%%%%%%%%%%%%%%%%%%%%%%%%
% Short Sectioned Assignment
% LaTeX Template
% Version 1.0 (5/5/12)
%
% This template has been downloaded from:
% http://www.LaTeXTemplates.com
%
% Original author:
% Frits Wenneker (http://www.howtotex.com)
%
% License:
% CC BY-NC-SA 3.0 (http://creativecommons.org/licenses/by-nc-sa/3.0/)
%
%%%%%%%%%%%%%%%%%%%%%%%%%%%%%%%%%%%%%%%%%
\documentclass[paper=letter, fontsize=11pt]{scrartcl}
\usepackage[spanish,es-noquoting]{babel}
%\usepackage[english]{babel}
\usepackage[T1]{fontenc} % Use 8-bit encoding that has 256 glyphs
\usepackage[utf8]{inputenc}
\usepackage{fourier} % Use the Adobe Utopia font for the document

\usepackage{amsmath,amsfonts,amsthm} % Math packages

\usepackage{sectsty} % Allows customizing section commands
\allsectionsfont{\normalfont\scshape} % Make all sections centered,
                                                 % the default font and small
                                                 % caps
\usepackage[american]{circuitikz}
\usetikzlibrary{matrix}

\usepackage{subfig}
\usepackage{fancyhdr} % Custom headers and footers
\usepackage{enumitem}
\setlist{nolistsep}

\usepackage[pdftex]{hyperref}
\hypersetup{
    colorlinks=true,%
    citecolor=black,%
    filecolor=black,%
    linkcolor=black,%
    urlcolor=blue
}

\pagestyle{fancyplain} % Makes all pages in the document conform to the custom
                       % headers and footers
\fancyhead{} % No page header - if you want one, create it in the same way as
             % the footers below
\fancyfoot[L]{} % Empty left footer
\fancyfoot[C]{} % Empty center footer
\fancyfoot[R]{\thepage} % Page numbering for right footer
\renewcommand{\headrulewidth}{0pt} % Remove header underlines
\renewcommand{\footrulewidth}{0pt} % Remove footer underlines
\setlength{\headheight}{13.6pt} % Customize the height of the header

% \numberwithin{equation}{section} % Number equations within sections (i.e. 1.1,
%                                  % 1.2, 2.1, 2.2 instead of 1, 2, 3, 4)
% \numberwithin{figure}{section} % Number figures within sections (i.e. 1.1, 1.2,
%                                % 2.1, 2.2 instead of 1, 2, 3, 4)
% \numberwithin{table}{section} % Number tables within sections (i.e. 1.1, 1.2,
%                               % 2.1, 2.2 instead of 1, 2, 3, 4)


\newcommand{\horrule}[1]{\rule{\linewidth}{#1}} % Create horizontal rule
                                                % command with 1 argument of
                                                % height

\title{
\normalfont \normalsize
\textsc{Universidad de Valparaíso, Departamento de Ingeniería Biomédica} \\
[25pt]
%\horrule{0.5pt} \\[0.4cm] % Thin top horizontal rule
\huge Teoría de Redes \\ % The assignment title
\LARGE Señales y Formas de Onda
%\horrule{2pt} \\[0.5cm] % Thick bottom horizontal rule
}

\author{Alejandro Weinstein}

\date{\normalsize 1\ts{er} semestre 2013} % Today's date or a custom date

\input{../macros.tex}

\begin{document}

\maketitle

\section{Algunas señales elementales}

\begin{defi}[Señal periódica]

La señal $v(t)$ es periódica con período $T$ si
%
\begin{equation}
\label{eq:periodic}
  v(t) = v(t + T) \text{ para todo } t.
\end{equation}

\end{defi}

Si $v(t)$ es periódica, entonces $v(t) = v(t + kT), k \in \mathbb{Z}$, para
todo $t$.

\begin{defi}[Señal sinusoidal]
  Una señal sinusoidal está dada por
  %
  \begin{equation}
    v(t) = A \cos( \omega t - \phi),
  \end{equation}
%
  donde $A$ es la amplitud, $\omega$ es la frecuencia angular, medida en
  radianes por segundo, y $\phi$ es el ángulo de fase, medido en radianes.
\end{defi}

La Figura \ref{fig:sinusoid} muestra la forma de onda correspondiente.

La frecuencia angular $\omega$ se relaciona con la frecuencia $f$, medida en
\emph{Hertz}, y el período de la sinusoide $T$, medido en segundos, a través de
las siguientes relaciones:
%
\begin{equation}
  \omega = 2 \pi f = \frac{2\pi}{T}.
\end{equation}

El ángulo de fase $\phi$ se relaciona con el retardo de tiempo $\tau$, medido
en segundos, a través de la relación
%
\begin{equation}
  \tau = \frac{\phi}{\omega}.
\end{equation}

\begin{figure}[h!]
  \centering
  \begin{tikzpicture}[xscale=3, yscale=1.2]
    \draw [->] (-0.5,0) -- (1.7,0);
    \draw [->] (0,-1.5) -- (0,1.5);
    \draw (0,1.5) node[above] {$v(t)$};
    \draw (1.7,0) node[above] {$t$};
    \draw [dotted](-0.5,1) -- (1.5,1);
    \draw [dotted](-0.5,-1) -- (1.5,-1);
    \draw [blue, ultra thick, domain=-0.5:1.5, samples=100] plot
    (\x, {cos(2*pi*\x r - 60)});
    \draw (0.03,1) node[above] {$A$};
    \draw (0.1,-1) node[below] {$-A$};
    \draw (0,-0.2)[<->, dotted] -- (0.167,-0.2);
    \draw (0.167,-0.2)[dotted] -- (0.167,1);
    \draw (0.08,-0.2) node[below] {$\tau$};
  \end{tikzpicture}
  \caption{Señal sinusoidal $v(t) = A \cos(\omega t - \phi)$. El retardo de
    tiempo está dado por $\tau = \frac{\phi}{\omega}$ segundos.}
\label{fig:sinusoid}
\end{figure}

\begin{defi}[Escalón unitario] La función escalón unitario, también conocida
  como función Heaviside, está definida como
  \begin{equation}
    u(t) =
    \begin{cases}
       0 & t < 0, \\
       1 & t > 0.
    \end{cases}
  \end{equation}
  %
  La Figura~\ref{fig:step} muestra la forma de onda correspondiente.
\end{defi}

Existen distintas convenciones para el valor que toma la función en
$t=0$. Algunos autores lo definen como $u(0) = \frac{1}{2}$, mientras que otros
lo hacen como $u(0)=1$. También se puede considerar que $u(0)$ es
indeterminado. En lo que sigue del curso esta distinción no es relevante.

\begin{figure}[h!]
  \centering
  \begin{tikzpicture}[xscale=2, yscale=1.2]
    \draw [->] (-2,0) -- (2,0); % x-axis
    \draw [->] (0,-0.5) -- (0,1.5); % y-axis
    \draw (0,1.5) node[above] {$u(t)$};
    \draw (2,0) node[above] {$t$};
    \draw [blue, ultra thick] (-2,0) -- (0,0)
    to(0,1)  to(2,1);
    \draw (0,1) node[left] {1};
  \end{tikzpicture}
  \caption{Escalón unitario $u(t)$.}
\label{fig:step}
\end{figure}

\begin{defi}[Rampa unitaria] La función rampa unitaria está definida como
  \begin{equation}
    r(t) =
    \begin{cases}
       0 & t < 0, \\
       t & t > 0.
    \end{cases}
  \end{equation}
  %
  La Figura~\ref{fig:step} muestra la forma de onda correspondiente.
\end{defi}

\begin{figure}[h!]
  \centering
  \begin{tikzpicture}[xscale=2, yscale=1.2]
    \draw [->] (-1.5,0) -- (1.5,0); % x-axis
    \draw [->] (0,-0.5) -- (0,1.5); % y-axis
    \draw (0,1.5) node[above] {$r(t)$};
    \draw (1.5,0) node[above] {$t$};
    \draw [blue, ultra thick] (-2,0) -- (0,0)
    to(1.5,1.5);
    \draw (0,1) node[left] {1} (1,0) node[below] {1};
    \draw [dotted](0,1) -- (1,1) -- (1,0);
  \end{tikzpicture}
  \caption{Rampa unitaria $r(t)$.}
\label{fig:rampa}
\end{figure}


\begin{defi}[Delta Dirac] El delta Dirac está definido como
  %
  \begin{equation}
    \delta(t) = \lim_{T \to 0} \delta_T(t),
  \end{equation}
\end{defi}
%
donde $\delta_T(t)$ está dado por
%
\begin{equation}
  \delta_T(t) =
  \begin{cases}
    0 & t < 0,\\
    \frac{1}{T} & 0 \leq t < T,\\
    0 & T \geq 0.
  \end{cases}
\end{equation}

\begin{figure}[h!]
  \centering
  \begin{tikzpicture}[xscale=2, yscale=1.2]
    \matrix[column sep=1cm]{
    \draw [->] (-0.5,0) -- (1.6,0); % x-axis
    \draw [->] (0,-0.5) -- (0,2.4); % y-axis
    \draw (0,2.4) node[above] {$\delta_T(t)$};
    \draw (1.6,0) node[above] {$t$};
    \draw [blue, ultra thick] (-0.5,0) -- (0,0)
    to(0,1) to (1,1) to (1,0) to (1.5,0);
    \draw (0,1) node[left] {$\frac{1}{T_1}$} (1,0) node[below] {$T_1$};
    \draw [red, thick] (-0.5,0) -- (0,0)
    to(0,2) to (0.5,2) to (0.5,0) to (1.5,0);
    \draw (0,2) node[left] {$\frac{1}{T_2}$} (0.5,0) node[below] {$T_2$}; &
    %
    \draw [->] (-0.5,0) -- (0.5,0); % x-axis
    \draw [->] (0,-0.5) -- (0,1.5); % y-axis
    \draw (0,1.5) node[above] {$\delta(t)$};
    \draw (0.5,0) node[above] {$t$};
    \draw [->, blue, ultra thick](0,0) -- (0,1);
    \draw (0,1) node[left] {1}; \\
    };
  \end{tikzpicture}
  \caption{Pulso $\delta_T(t)$ para $T_2 < T_1$ y delta Dirac $\delta(t)$
    simbolizado por una flecha de largo unitario.}
\label{fig:dirac}
\end{figure}

La Figura~\ref{fig:dirac} muestra la función $\delta_T(t)$ para dos valores
distintos de $T$ $T_1$ y $T_2$, con $T_2 < T_1$. El delta Dirac se simboliza
con una flecha de largo unitario localizada en $t=0$, como lo muestra la misma
figura.

El delta Dirac satisface las siguientes propiedades:
%
\begin{gather}
  \delta(t) = 0 \text{ para todo } t \neq 0 \\
  \int_{-\infty}^\infty \delta(t) \ dt = 1 \\
  \int_{-\infty}^\infty \delta(t - \tau) v(t) \ dt = v(\tau).
\end{gather}

El delta Dirac, el escalón unitario y la rampa unitaria están relacionados por
las expresiones que siguen.
%
\begin{equation*}
  u(t) = \int_{-\infty}^t \delta(x) \ dx; \quad
  r(t) = \int_{-\infty}^t u(x) \ dx.
\end{equation*}

\begin{defi}[Exponencial]

La función exponencial está definida como
%
\begin{equation}
  v(t) = e^{at} \quad a \in \real.
\end{equation}
%
La Figura~\ref{fig:exp} muestra la forma de onda correspondiente.
\end{defi}

\begin{figure}[h!]
  \centering
  \begin{tikzpicture}[xscale=2, yscale=0.4]
    \draw [->] (-1,0) -- (2,0); % x-axis
    \draw [->] (0,-0.5) -- (0,8); % y-axis
    \draw (0,8) node[right] {$v(t)$};
    \draw (2,0) node[above] {$t$};
    \draw [blue, ultra thick, domain=-1:2, samples=100] plot
    (\x, {exp(\x)});
  \end{tikzpicture}
  \caption{Función exponencial $e^{at}$.}
\label{fig:exp}
\end{figure}

Una de las propiedades fundamentales de las exponenciales es que sus derivadas
y antiderivadas son también una función exponencial:
%
\begin{equation}
  \frac{dv}{dt} = a e^{at}, \quad \int e^{at} = \frac{1}{a} e^{at}.
\end{equation}

Cuando el parámetro $a$ es negativo, la exponencial es decreciente, \ie, su
valor disminuye en la medida que el tiempo aumenta. En este caso comúnmente la
exponencial se define como
%
\begin{equation}
  v(t) = e^{-t/\tau} \quad \tau > 0.
\end{equation}
%
El parámetro $\tau$ se conoce como \emph{constante de tiempo}. La
Figura~\ref{fig:expDec} muestra la forma de onda correspondiente.

Las siguientes son algunas propiedades importantes de la función exponencial
decreciente.
%
\begin{gather}
  v(0) = 1 \\
  \lim_{t \to \infty} v(t) = 0 \\
  v(4\tau) = 0.018 \approx 0.
\end{gather}

\begin{figure}[h!]
  \centering
  \begin{tikzpicture}[xscale=1.5, yscale=2]
    \draw [->] (-1,0) -- (4.1,0); % x-axis
    \draw [->] (0,-0.5) -- (0,3); % y-axis
    \draw (0,3) node[right] {$v(t)$};
    \draw (4.1,0) node[right] {$t$};
    \draw [blue, ultra thick, domain=-1:4, samples=100] plot
    (\x, {exp(-\x)});
  \end{tikzpicture}
  \caption{Función exponencial decreciente $e^{-t/\tau}$.}
\label{fig:expDec}
\end{figure}

Comúnmente en el análisis de circuitos eléctricos la función exponencial
aparece en la forma
%
\begin{equation}
  v(t) = V_F + (V_I - V_F) e^{-t/\tau}, \ t \ge 0,
\end{equation}
%
donde $v(0)=V_I$ es el valor inicial, y $\lim_{t \to \infty} v(t) = V_F$ es el
valor final, también conocido como valor en estado estacionario. La
Figura~\ref{fig:expGen} muestra la forma de onda correspondiente.

\begin{figure}[h!]
  \centering
  \begin{tikzpicture}[xscale=1.5, yscale=2]
    \draw [->] (-0.1,0) -- (5.1,0); % x-axis
    \draw [->] (0,-0.5) -- (0,2.3); % y-axis
    \draw (0,2.3) node[right] {$v(t)$};
    \draw (5.1,0) node[right] {$t$};
    \draw [blue, ultra thick, domain=0:5, samples=100] plot
    (\x, {2 - 1.5 * exp(-\x)});
    \draw (0,2) node[left] {$V_F$} (0,0.5) node[left] {$V_I$};
    \draw [dotted] (0,2) -- (5,2);
  \end{tikzpicture}
  \caption{Función $v(t) = V_F + (V_I - V_F) e^{-t/\tau}, \ t \ge 0$ para $t>0$.}
\label{fig:expGen}
\end{figure}

\section{Valor medio y efectivo}

\begin{defi}[Valor medio]
  Dado $v(t)$ periódico con período $T$, el valor medio se define como
  %
  \begin{equation}
    \label{eq:vmedio}
    v_{\text{medio}}  = \frac{1}{T} \int_0^T v(t) \ dt.
  \end{equation}
\end{defi}

El valor medio de $v(t)$ también se denota como $\bar{v}$ o $\langle v(t)
\rangle$. Como $v(t)$ es periódico, los rangos de la integral en la ecuación
\eqref{eq:vmedio} se pueden desplazar en un valor arbitrario $t_0 \in \real$:
%
\begin{equation}
  v_{\text{medio}}  = \frac{1}{T} \int_{t_0}^{t_0+T} v(t) \ dt.
\end{equation}

\begin{defi}[Valor efectivo]
  Dado $v(t)$ periódico con período $T$, el valor efectivo se define como
  %
  \begin{equation}
    \label{eq:vmedio}
    v_{\text{ef}}  = \sqrt{\frac{1}{T} \int_0^T v^2(t) \ dt}.
  \end{equation}
  %
\end{defi}

El valor efectivo también se conoce como \emph{valor eficaz}, o por su nombre
en inglés \emph{Root Mean Square (RMS) value}, y también se denota como
$v_{\text{RMS}}$.

\begin{ex}[Valor medio de una sinusoide]
  Calcule el valor medio de $v(t) = A cos(\omega t)$.

  En este caso el período de $v(t)$ es $T = \frac{2\pi}{\omega}$. El valor
  medio es
  %
  \begin{equation*}
    v_{\text{medio}} = \frac{A \omega}{2\pi} \int_0^{\frac{2\pi}{\omega}} cos(\omega t)
      \ dt = \frac{A}{2\pi} sin(\omega t) \biggr|_0^{\frac{2\pi}{\omega}} = 0.
  \end{equation*}
\end{ex}

\begin{ex}[Valor eficaz de una sinusoide]
  Calcule el valor eficaz de $v(t) = A cos(\omega t)$.

  Usando la identidad trigonométrica $\cos^2(x) = \frac{1}{2}(1 + \cos(2x))$,
  el valor eficaz se calcula como
  \begin{equation*}
    \begin{split}
      v_{\text{RMS}} &= \sqrt{\frac{1}{T} \int_0^T A^2 \cos^2(\omega t) \ dt}\\
      &= A \sqrt{\frac{1}{T} \int_0^T \frac{1}{2} \ dt +
        \frac{1}{T} \int_0^T \cos(2\omega t) \ dt}\\
      &= A \sqrt{\frac{1}{2T} T + 0} \\
      &= \frac{A}{\sqrt{2}}.
    \end{split}
  \end{equation*}
\end{ex}

El valor eficaz es útil para calcular la potencia media disipada por una
resistencia cuando el voltaje entre sus terminales es una función periódica.
Dado el voltaje periódico $v(t)$ entre los terminales de una resistencia de
valor $R$, la potencia instantánea disipada por la resistencia es
%
\begin{equation}
  p(t) = \frac{v^2(t)}{R}.
\end{equation}
%
El valor medio de la potencia instantánea es
%
\begin{equation*}
  \begin{split}
    p_{\text{medio}} &= \frac{1}{T} \int_0^T p(t) \ dt \\
    &= \frac{1}{R T} \int_0^T v^2(t) \ dt \\
    &= \frac{v^2_{RMS}}{R}.
  \end{split}
\end{equation*}
%
Usando el mismo razonamiento, se concluye que si una corriente periódica $i(t)$
pasa por una resistencia de valor $R$, la potencia disipada por la resistencia
es
%
\begin{equation*}
   p_{\text{medio}} = i^2_{RMS} R.
 \end{equation*}

\section{Traslación y dilatación}

\begin{figure}[ht!]
  \centering
  \begin{tikzpicture}[xscale=1.5, yscale=1.1]
  \matrix[column sep=0.5cm]{
    \draw [->] (-2.1,0) -- (2.1,0); % x-axis
    \draw [->] (0,-0.5) -- (0,1.5); % y-axis
    \draw (0,1.5) node[right] {$v(t)$};
    \draw (2.1,0) node[right] {$t$};
    \draw [blue, ultra thick] (-2,0) -- (-0.5,0)
    -- (0,1) -- (0.5,0) -- (2,0); &
    %
    \draw [->] (-2.1,0) -- (2.1,0); % x-axis
    \draw [->] (0,-0.5) -- (0,1.5); % y-axis
    \draw (0,1.5) node[right] {$v(t-T)$};
    \draw (2.1,0) node[right] {$t$};
    \draw [blue, ultra thick] (-2,0) -- (0,0)
    -- (0.5,1) -- (1,0) -- (2.0,0); \\
    %
        \draw [->] (-2.1,0) -- (2.1,0); % x-axis
    \draw [->] (0,-0.5) -- (0,1.5); % y-axis
    \draw (0,1.5) node[right] {$v(t/a)$};
    \draw (2.1,0) node[right] {$t$};
    \draw [blue, ultra thick] (-2,0) -- (-1,0)
    -- (0,1) -- (1,0) -- (2,0); &
    %
    \draw [->] (-2.1,0) -- (2.1,0); % x-axis
    \draw [->] (0,-0.5) -- (0,1.5); % y-axis
    \draw (0,1.5) node[right] {$v(\frac{t-T}{a})$};
    \draw (2.1,0) node[right] {$t$};
    \draw [blue, ultra thick] (-2,0) -- (-0.5,0)
    -- (0.5,1) -- (1.5,0) -- (2,0); \\
    };
  \end{tikzpicture}
  \caption{Traslación y dilatación de $v(t)$.}
\label{fig:trans}
\end{figure}

Dada la señal $v(t)$ y el parámetro $T > 0$ medido en segundos, $v(t-T)$ es una
señal con la misma forma que $v(t)$, pero trasladada $T$ segundos hacia la
derecha en el eje del tiempo. Podemos decir que $v(t-T)$ es igual a $v(t)$
retardada $T$ segundos. Análogamente, $v(t + T)$ es una señal con la misma
forma que $v(t)$, pero trasladad $T$ segundos hacia la izquierda en el eje del
tiempo. Podemos decir que $v(t-T)$ es igual a $v(t)$ adelantada $T$ segundos.


Dada la señal $v(t)$ y el parámetro $a > 1$ medido en segundos, $v(t/a)$
corresponde a una versión dilatada en el tiempo de $v(t)$, y $v(at)$
corresponde a una versión comprimida en el tiempo de $v(t)$.


La Figura~\ref{fig:trans} muestra un ejemplo de translación, de dilatación y de
la combinación de la translación y dilatación.



\end{document}

%%% Local Variables:
%%% mode: latex
%%% TeX-master: t
%%% Local IspellDict: castellano
%%% End:
